\documentclass[a4paper,11pt]{article}

% ── Codificación y idioma ──
\usepackage[utf8]{inputenc}
\usepackage[T1]{fontenc}
\usepackage[spanish]{babel}

% ── Diseño de página ──
\usepackage[margin=2.2cm]{geometry}

% ── Tipografía ──
\usepackage{lmodern}

% ── Imágenes ──
\usepackage{graphicx}
\graphicspath{{images/}}

% ── Tablas y listas ──
\usepackage{booktabs}
\usepackage{array}
\usepackage{tabularx}
\usepackage{longtable}
\usepackage{enumitem}
\usepackage{multicol}
\usepackage{float}

% ── Símbolos ──
\usepackage{marvosym}   % para \Lightning

% ── Colores y cajas ──
\usepackage[table,dvipsnames]{xcolor}
\usepackage{tcolorbox}
\tcbuselibrary{skins, breakable}

% ── Títulos ──
\usepackage{titlesec}
\titleformat{\section}{\normalfont\Large\bfseries\color{MidnightBlue}}{\thesection}{1em}{}[\titlerule[0.8pt]]
\titleformat{\subsection}{\normalfont\large\bfseries\color{RoyalBlue}}{\thesubsection}{1em}{}

% ── Hipervínculos ──
\usepackage{hyperref}
\hypersetup{colorlinks=true, linkcolor=MidnightBlue, urlcolor=RoyalBlue}

% ── Encabezados y pies de página ──
\usepackage{fancyhdr}
\pagestyle{fancy}
\fancyhf{}
\fancyhead[L]{\small\textcolor{gray}{Tema 5 --- Conectores y Cableado (Resumen)}}
\fancyhead[R]{\small\textcolor{gray}{MME --- Curso 2024--2025}}
\fancyfoot[C]{\thepage}
\renewcommand{\headrulewidth}{0.4pt}

% ── Cajas personalizadas ──
\newtcolorbox{infobox}[1][]{
  colback=blue!5, colframe=MidnightBlue, fonttitle=\bfseries,
  title=#1, breakable, boxrule=0.5pt, arc=2mm
}
\newtcolorbox{warnbox}[1][]{
  colback=orange!8, colframe=BurntOrange, fonttitle=\bfseries,
  title=#1, breakable, boxrule=0.5pt, arc=2mm
}
\newtcolorbox{keybox}[1][]{
  colback=green!5, colframe=OliveGreen, fonttitle=\bfseries,
  title=#1, breakable, boxrule=0.5pt, arc=2mm
}

% ══════════════════════════════════════════════════════════════════
\title{%
  \textbf{Tema 5: Conectores y Cableado}\\[4pt]
  \Large Resumen orientado a conectores actuales\\[6pt]
  \large Montaje y Mantenimiento de Equipos\\
  Ciclo Formativo de Grado Medio --- Curso 2024--2025\\
  \normalsize I.E.S.\ Valle del Jerte
}
\author{}
\date{}

\begin{document}
\maketitle
\thispagestyle{fancy}

% ══════════════════════════════════════════════════════════════════
\begin{keybox}[Objetivo de este resumen]
Este documento condensa el Tema~5 centrándose en los \textbf{conectores y cableados que se utilizan actualmente} en equipos informáticos. Los conectores obsoletos (puerto serie, paralelo, PS/2, VGA, DVI, IDE, FireWire\ldots) se mencionan brevemente en la sección final a modo de contexto histórico, pero el estudio se enfoca en las interfaces que encontrarás montando y manteniendo equipos hoy en día.
\end{keybox}

\tableofcontents
\newpage

% ══════════════════════════════════════════════════════════════════
\section{Conceptos básicos de conectividad}
% ══════════════════════════════════════════════════════════════════

\subsection{Pines y contactos}
La conexión entre un puerto (en la placa base o tarjeta de expansión) y un conector (en el cable) se realiza mediante \textbf{pines}: patillas metálicas que establecen contacto eléctrico. Cada pin transporta una señal concreta (datos, alimentación, masa, reloj, etc.).

\subsection{Formatos macho y hembra}
\begin{itemize}[nosep]
  \item \textbf{Macho (M):} los pines sobresalen y se insertan en la parte hembra.
  \item \textbf{Hembra (F):} dispone de orificios que reciben los pines macho.
\end{itemize}
Por regla general, los \emph{puertos} de los equipos son hembra y los \emph{conectores} de los cables son macho, aunque hay excepciones (por ejemplo, el puerto serie COM es macho).

\subsection{Alargadores, adaptadores y hubs}
\begin{itemize}[nosep]
  \item \textbf{Alargador:} cable con un extremo macho y otro hembra del mismo tipo; extiende el alcance de un puerto.
  \item \textbf{Adaptador:} transforma un tipo de conector en otro (p.~ej.\ DisplayPort a HDMI). Puede ser un dongle o llevar cable.
  \item \textbf{Hub:} un único cable que ofrece varios puertos en el otro extremo (p.~ej.\ hub USB-C con HDMI + USB-A + Ethernet).
\end{itemize}

\subsection{Familias clásicas de conectores}

Antes de estudiar los conectores actuales, conviene conocer las familias de conectores que han existido en informática, ya que su nomenclatura sigue apareciendo:

\begin{figure}[htbp]
  \centering
  \includegraphics[width=0.55\textwidth]{conectoresDIN}
  \caption{Conectores DIN: de 3 a 8 pines, diámetro de 13,2~mm. Origen en la estandarización alemana.}
  \label{fig:din}
\end{figure}

\begin{figure}[htbp]
  \centering
  \includegraphics[width=0.55\textwidth]{conectoresminidin}
  \caption{Conectores Mini-DIN: versión reducida (9,5~mm), de 3 a 9 pines. El Mini-DIN~6 se utilizó para PS/2.}
  \label{fig:minidin}
\end{figure}

\begin{figure}[htbp]
  \centering
  \includegraphics[width=0.55\textwidth]{conectoresDSUB}
  \caption{Conectores D-Sub: carcasa metálica en forma de D con pines en filas. VGA (DE-15), serie (DE-9) y paralelo (DB-25) pertenecen a esta familia.}
  \label{fig:dsub}
\end{figure}


% ══════════════════════════════════════════════════════════════════
\section{USB (Universal Serial Bus)}
\label{sec:usb}
% ══════════════════════════════════════════════════════════════════

El USB es, con diferencia, el \textbf{estándar de conectividad más importante en la actualidad}. Permite conectar prácticamente cualquier periférico: ratones, teclados, discos externos, impresoras, cámaras, smartphones, monitores\ldots{} y además puede proporcionar alimentación eléctrica.

\subsection{Características clave del USB}
\begin{itemize}
  \item \textbf{Plug-and-Play:} el sistema operativo reconoce el dispositivo al conectarlo sin reiniciar (conexión \emph{en caliente}).
  \item \textbf{Alimentación eléctrica:} proporciona energía a dispositivos de bajo consumo directamente por el cable, y con \textbf{USB Power Delivery (USB PD)} puede suministrar hasta 240~W (estándar USB PD 3.1).
  \item \textbf{Topología en estrella:} mediante hubs se pueden conectar hasta 127 dispositivos en un mismo controlador.
  \item \textbf{Compatibilidad hacia atrás:} un dispositivo USB~2.0 funciona en un puerto USB~3.x (a velocidad 2.0).
\end{itemize}

\subsection{Versiones y velocidades del USB}

\begin{infobox}[Nota sobre la nomenclatura]
El USB Implementers Forum (USB-IF) ha renombrado los estándares varias veces, generando confusión. En la tabla se incluyen los nombres más comunes y el nombre de marketing actual.
\end{infobox}

\begin{center}
\renewcommand{\arraystretch}{1.3}
\begin{tabularx}{\textwidth}{>{\bfseries}l c c X}
\toprule
\textbf{Estándar} & \textbf{Nombre marketing} & \textbf{Velocidad máx.} & \textbf{Notas} \\
\midrule
USB 2.0        & USB 2.0        & 480 Mbps (60 MB/s)     & Aún presente en ratones, teclados y hubs baratos. \\
USB 3.0 / 3.1 Gen~1 / 3.2 Gen~1 & USB 5Gbps & 5 Gbps (625 MB/s) & Puerto identificable por el \textbf{color azul} en USB-A. \\
USB 3.1 Gen~2 / 3.2 Gen~2       & USB 10Gbps & 10 Gbps (1,25 GB/s) & Común en puertos USB-C de gama media-alta. \\
USB 3.2 Gen~2x2                  & USB 20Gbps & 20 Gbps (2,5 GB/s) & Solo disponible en USB-C (usa dos carriles de datos). \\
USB4 Gen~3x2  & USB4 40Gbps    & 40 Gbps (5 GB/s)       & Basado en Thunderbolt~3. Solo USB-C. \\
USB4 v2.0     & USB4 80Gbps    & 80 Gbps (10 GB/s)      & La especificación más reciente (2022). Solo USB-C. \\
\bottomrule
\end{tabularx}
\end{center}

\subsection{Tipos de conectores USB}

\begin{figure}[htbp]
  \centering
  \includegraphics[width=0.7\textwidth]{ConectoresUSB}
  \caption{Familia de conectores USB: Tipo-A, Tipo-B, Mini-USB, Micro-USB y USB-C en sus diferentes versiones.}
  \label{fig:usb_conectores}
\end{figure}

\begin{center}
\renewcommand{\arraystretch}{1.3}
\begin{tabularx}{\textwidth}{>{\bfseries}l c X}
\toprule
\textbf{Conector} & \textbf{Pines} & \textbf{Uso actual} \\
\midrule
USB Tipo-A       & 4 (2.0) / 9 (3.x) & Rectangular, no reversible. Sigue siendo el más común en PCs de sobremesa, portátiles y cargadores. Color azul = USB 3.x. \\
USB Tipo-B       & 4 / 9             & Cuadrado con bordes biselados. Casi solo en impresoras y algún periférico profesional (interfaces de audio, etc.). En desuso progresivo. \\
Micro-USB        & 5                 & Trapezoidal pequeño. Fue estándar en smartphones Android hasta $\sim$2018. Aún en dispositivos baratos (mandos, auriculares económicos). \\
\textbf{USB Tipo-C}   & \textbf{24}  & \textbf{Conector moderno, ovalado y reversible.} Es el presente y futuro: datos, vídeo (DisplayPort Alt Mode), carga (USB PD) y audio, todo en uno. Obligatorio en la UE desde 2024 para dispositivos móviles. \\
\bottomrule
\end{tabularx}
\end{center}

\begin{figure}[htbp]
  \centering
  \includegraphics[width=0.7\textwidth]{usbs}
  \caption{Puertos USB en el panel trasero de una placa base: USB 2.0 (negro), USB 3.0 (azul) y USB-C.}
  \label{fig:usb_puertos}
\end{figure}

\begin{warnbox}[Cuidado: aspecto $\neq$ velocidad]
Un cable con conector USB-C no implica que sea USB~3.x o USB4. El \textbf{tipo de conector} (forma física) es independiente del \textbf{estándar/velocidad}. Existen cables USB-C que solo soportan USB~2.0 (480~Mbps). Es fundamental verificar las especificaciones del cable y del puerto.
\end{warnbox}

\subsection{Thunderbolt: la extensión premium del USB-C}
\label{sec:thunderbolt}

Thunderbolt es una tecnología desarrollada por \textbf{Intel y Apple} que utiliza el \textbf{mismo conector USB-C} pero ofrece prestaciones superiores.

\begin{center}
\renewcommand{\arraystretch}{1.3}
\begin{tabularx}{\textwidth}{>{\bfseries}l c c X}
\toprule
\textbf{Versión} & \textbf{Velocidad} & \textbf{Vídeo} & \textbf{Características} \\
\midrule
Thunderbolt 3 & 40 Gbps & 2$\times$4K a 60Hz & Compatible con USB-C. Carga hasta 100~W. \\
Thunderbolt 4 & 40 Gbps & 2$\times$4K a 60Hz & Misma velocidad, pero \textbf{requisitos mínimos más exigentes} (certificación obligatoria). Soporte para eGPU y daisy-chain de hasta 6 dispositivos. \\
Thunderbolt 5 & 80/120 Gbps & 3$\times$4K a 144Hz & Lanzado en 2024. Usa PAM-3 para alcanzar 120~Gbps en modo asimétrico (útil para vídeo). \\
\bottomrule
\end{tabularx}
\end{center}

Un puerto Thunderbolt se identifica por el \textbf{símbolo del rayo (\Lightning)} junto al conector USB-C.

\textbf{Aplicaciones de Thunderbolt:}
\begin{itemize}[nosep]
  \item Conexión de \textbf{eGPU} (tarjetas gráficas externas) para portátiles.
  \item \textbf{Docking stations:} un solo cable USB-C conecta el portátil a monitores, red, periféricos y carga.
  \item Transferencia ultrarrápida a \textbf{almacenamiento externo} (RAID, NVMe externo).
  \item Conexión en \textbf{daisy-chain} de múltiples monitores y dispositivos.
\end{itemize}


% ══════════════════════════════════════════════════════════════════
\section{Conectores de vídeo: HDMI y DisplayPort}
\label{sec:video}
% ══════════════════════════════════════════════════════════════════

Los dos estándares de vídeo que se utilizan actualmente son \textbf{HDMI} y \textbf{DisplayPort}. Ambos transmiten audio y vídeo digital por un solo cable.

\subsection{HDMI (High-Definition Multimedia Interface)}

Introducido en 2003, HDMI se diseñó para unificar audio y vídeo en un solo cable, sustituyendo al euroconector, VGA y los cables RCA. Es el estándar dominante en \textbf{electrónica de consumo}: televisores, consolas, reproductores Blu-ray, barras de sonido, proyectores\ldots

\subsubsection{Tipos de conectores HDMI}

\begin{figure}[htbp]
  \centering
  \includegraphics[width=0.6\textwidth]{ConectorHDMI}
  \caption{Tipos de conectores HDMI: Tipo~A (estándar), Mini-HDMI (Tipo~C) y Micro-HDMI (Tipo~D).}
  \label{fig:hdmi_conectores}
\end{figure}

\begin{itemize}[nosep]
  \item \textbf{HDMI Tipo~A (estándar):} 19 pines. El más común en TVs, monitores y tarjetas gráficas.
  \item \textbf{HDMI Mini (Tipo~C):} 19 pines, más pequeño. Usado en cámaras y algunas tablets.
  \item \textbf{HDMI Micro (Tipo~D):} 19 pines, aún menor. Dispositivos ultracompactos (cámaras de acción, drones).
\end{itemize}

\subsubsection{Evolución de versiones HDMI}

\begin{center}
\renewcommand{\arraystretch}{1.3}
\begin{tabularx}{\textwidth}{>{\bfseries}l c c X}
\toprule
\textbf{Versión} & \textbf{Ancho de banda} & \textbf{Resolución máx.} & \textbf{Novedades principales} \\
\midrule
HDMI 1.4 & 10,2 Gbps & 4K a 30 Hz  & Introduce soporte básico para 4K y 3D. Canal Ethernet (HEC). \\
HDMI 2.0 & 18 Gbps   & 4K a 60 Hz  & Soporte para HDR. Hasta 32 canales de audio. \\
HDMI 2.1 & 48 Gbps   & \textbf{8K a 60 Hz} / 4K a 120 Hz & VRR (frecuencia variable), ALLM (modo baja latencia), eARC (audio de alta calidad). Es la \textbf{versión actual}. \\
\bottomrule
\end{tabularx}
\end{center}

\begin{infobox}[Funciones destacadas de HDMI 2.1]
\begin{itemize}[nosep]
  \item \textbf{VRR (Variable Refresh Rate):} sincroniza la tasa de refresco del monitor con los fotogramas de la GPU, eliminando el \emph{tearing}. Esencial en gaming.
  \item \textbf{ALLM (Auto Low Latency Mode):} el dispositivo fuente indica al TV que active el modo de baja latencia automáticamente.
  \item \textbf{eARC (Enhanced Audio Return Channel):} permite enviar audio de alta calidad (Dolby Atmos, DTS:X sin compresión) desde el TV a una barra de sonido o receptor AV por el mismo cable HDMI.
  \item \textbf{Soporte HDR:} HDR10, HDR10+, Dolby Vision, HLG. Mejoran el rango dinámico y la fidelidad de color.
\end{itemize}
\end{infobox}

\subsection{DisplayPort}

Introducido en 2006 por VESA, DisplayPort es el estándar preferido en \textbf{entornos PC}, monitores de alta gama y gaming profesional. A diferencia de HDMI (que requiere licencia), DisplayPort es un estándar libre de royalties.

\subsubsection{Tipos de conectores DisplayPort}

\begin{figure}[htbp]
  \centering
  \includegraphics[width=0.6\textwidth]{displayport}
  \caption{Conector DisplayPort estándar (izq.) y Mini DisplayPort (der.). Nótese el clip de retención del conector estándar.}
  \label{fig:displayport}
\end{figure}

\begin{itemize}[nosep]
  \item \textbf{DisplayPort estándar:} conector de 20 pines con un \textbf{clip de retención} que evita desconexiones accidentales. Presente en tarjetas gráficas y monitores.
  \item \textbf{Mini DisplayPort:} versión compacta. Fue popular en portátiles Apple y Surface de Microsoft. Hoy menos frecuente (desplazado por USB-C).
\end{itemize}

\subsubsection{Evolución de versiones DisplayPort}

\begin{center}
\renewcommand{\arraystretch}{1.3}
\begin{tabularx}{\textwidth}{>{\bfseries}l c c X}
\toprule
\textbf{Versión} & \textbf{Ancho de banda} & \textbf{Resolución máx.} & \textbf{Novedades} \\
\midrule
DP 1.2 & 21,6 Gbps  & 4K a 60 Hz & Introduce MST (Multi-Stream Transport). \\
DP 1.4 & 32,4 Gbps  & 8K a 60 Hz (con DSC) & Soporte HDR10, DSC (compresión visual sin pérdida perceptible). \\
DP 2.0 & 77,4 Gbps  & 16K (con DSC) & Enorme salto de ancho de banda. \\
DP 2.1 & 77,4 Gbps  & 4K a 240 Hz & Refinamiento del 2.0, mejor gestión de cables y compatibilidad. Es la \textbf{versión actual}. \\
\bottomrule
\end{tabularx}
\end{center}

\begin{keybox}[Ventaja exclusiva de DisplayPort: MST (Multi-Stream Transport)]
MST permite conectar \textbf{varios monitores en cadena} (\emph{daisy-chain}) usando un solo puerto DisplayPort del PC. El primer monitor se conecta al PC, y de ese monitor sale otro cable DP al segundo monitor, y así sucesivamente. Esto reduce el número de cables y puertos necesarios en la tarjeta gráfica. HDMI \textbf{no soporta} esta función.
\end{keybox}

\subsection{HDMI vs DisplayPort: ¿cuándo usar cada uno?}

\begin{center}
\renewcommand{\arraystretch}{1.3}
\begin{tabularx}{\textwidth}{l X X}
\toprule
\textbf{Criterio} & \textbf{HDMI} & \textbf{DisplayPort} \\
\midrule
Uso principal & TV, consolas, home cinema & Monitores de PC, gaming en PC, workstations \\
Audio de retorno & Sí (ARC/eARC) & No \\
Multi-monitor (daisy-chain) & No & Sí (MST) \\
Adaptadores & DP$\to$HDMI fácil (pasivo) & HDMI$\to$DP difícil (activo, costoso) \\
Conector reversible & No & No (pero USB-C con DP Alt Mode sí) \\
Licencia & Sí (de pago para fabricantes) & Libre de royalties \\
\bottomrule
\end{tabularx}
\end{center}


% ══════════════════════════════════════════════════════════════════
\section{Conectores de audio: el puerto Jack}
\label{sec:audio}
% ══════════════════════════════════════════════════════════════════

Aunque las conexiones inalámbricas (Bluetooth) y digitales (USB-C) ganan terreno, el \textbf{Jack de 3,5~mm} sigue siendo el conector de audio analógico estándar en PCs, portátiles, equipos de sonido y auriculares con cable.

\subsection{Tamaños de conectores Jack}

\begin{figure}[htbp]
  \centering
  \includegraphics[width=0.55\textwidth]{ConectorJack}
  \caption{Comparativa de tamaños de conectores Jack: 6,35~mm (estándar), 3,5~mm (mini-jack) y 2,5~mm (sub-mini).}
  \label{fig:jack_tamanos}
\end{figure}

\begin{itemize}[nosep]
  \item \textbf{6,35~mm (Jack estándar):} instrumentos musicales, amplificadores, auriculares profesionales.
  \item \textbf{3,5~mm (Mini-Jack):} el más habitual. Ordenadores, smartphones (los que aún lo conservan), auriculares de consumo.
  \item \textbf{2,5~mm (Sub-mini-Jack):} muy raro hoy en día. Se usaba en teléfonos antiguos.
\end{itemize}

\subsection{TRS vs TRRS}
Los anillos aislantes del conector dividen la clavija en zonas, cada una con una señal:

\begin{figure}[htbp]
  \centering
  \includegraphics[width=0.5\textwidth]{jackmicro}
  \caption{Diferencia entre TRS (3 contactos, solo audio) y TRRS (4 contactos, audio + micrófono).}
  \label{fig:trs_trrs}
\end{figure}

\begin{itemize}[nosep]
  \item \textbf{TRS (Tip--Ring--Sleeve):} 3 contactos. Transporta audio estéreo (canal izquierdo + canal derecho + masa). Típico de auriculares sin micrófono.
  \item \textbf{TRRS (Tip--Ring--Ring--Sleeve):} 4 contactos. Añade un canal para el \textbf{micrófono}. Es el que llevan los auriculares con micrófono integrado (manos libres). Estándar en smartphones y portátiles con un único puerto combo.
\end{itemize}

\subsection{Códigos de colores en la placa base}
Los puertos Jack de la tarjeta de sonido integrada se identifican por colores:

\begin{center}
\renewcommand{\arraystretch}{1.25}
\begin{tabular}{>{\bfseries}l l l}
\toprule
\textbf{Color} & \textbf{Función} & \textbf{Tipo de señal} \\
\midrule
\textcolor{green!70!black}{Verde}   & Salida de altavoces / auriculares (frontal) & Salida estéreo \\
\textcolor{pink}{Rosa}              & Entrada de micrófono                       & Entrada mono \\
\textcolor{blue}{Azul}              & Entrada de línea (Line In)                 & Entrada estéreo \\
\textcolor{orange}{Naranja}         & Altavoz central / subwoofer                & Salida (5.1/7.1) \\
Negro                               & Altavoces traseros                         & Salida (5.1/7.1) \\
\textcolor{gray}{Gris}              & Altavoces laterales                        & Salida (7.1) \\
\bottomrule
\end{tabular}
\end{center}

\begin{figure}[htbp]
  \centering
  \includegraphics[width=0.5\textwidth]{MINIJACK}
  \caption{Puertos Mini-Jack en el panel trasero de una placa base, con el código de colores.}
  \label{fig:jack_colores}
\end{figure}


% ══════════════════════════════════════════════════════════════════
\section{Conectores de red}
\label{sec:red}
% ══════════════════════════════════════════════════════════════════

\subsection{RJ-45 (Ethernet)}

El conector \textbf{RJ-45 (8P8C)} es el estándar para redes cableadas Ethernet. Lo encontramos en tarjetas de red de PCs, routers, switches y rosetas de pared.

\begin{figure}[htbp]
  \centering
  \includegraphics[width=0.5\textwidth]{ConectorRJ45}
  \caption{Conector RJ-45 (8P8C) con sus 8 contactos y clip de retención.}
  \label{fig:rj45}
\end{figure}

\subsubsection{Características del puerto RJ-45}
\begin{itemize}[nosep]
  \item 8 pines / 8 conductores (8P8C).
  \item LEDs indicadores: \textbf{luz fija} = enlace establecido; \textbf{luz parpadeante} = tráfico de datos.
  \item Clip de retención para evitar desconexiones.
\end{itemize}

\subsubsection{Categorías de cable Ethernet}

La \textbf{categoría del cable} determina la velocidad máxima y la frecuencia soportada. Un error frecuente es asumir que ``cualquier cable de red vale''; usar una categoría inadecuada limita el rendimiento de toda la red.

\begin{center}
\renewcommand{\arraystretch}{1.3}
\begin{tabularx}{\textwidth}{>{\bfseries}l c c X}
\toprule
\textbf{Categoría} & \textbf{Frecuencia} & \textbf{Velocidad máx.} & \textbf{Uso recomendado} \\
\midrule
Cat 5e & 100 MHz  & 1 Gbps   & Redes domésticas básicas. Mínimo aceptable hoy. \\
Cat 6  & 250 MHz  & 1 Gbps (10 Gbps hasta 55 m)  & Redes domésticas y oficinas. \textbf{Recomendado como estándar.} \\
Cat 6a & 500 MHz  & 10 Gbps (hasta 100 m) & Redes profesionales, edición de vídeo en red, servidores. \\
Cat 7  & 600 MHz  & 10 Gbps  & Data centers. Cable apantallado (STP). Conector propio (GG45/TERA), aunque se crimpa con RJ-45 compatible. \\
Cat 8  & 2000 MHz & 25/40 Gbps & Data centers de alta densidad. Distancia corta (30 m). \\
\bottomrule
\end{tabularx}
\end{center}

\subsubsection{Tipos de cable según cruce}
\begin{itemize}[nosep]
  \item \textbf{Cable directo (straight-through):} conecta dispositivos \emph{diferentes} (PC $\leftrightarrow$ switch, PC $\leftrightarrow$ router). Es el 99\% de los cables que se usan.
  \item \textbf{Cable cruzado (crossover):} conecta dispositivos \emph{iguales} (PC $\leftrightarrow$ PC, switch $\leftrightarrow$ switch). Hoy en día casi innecesario porque los dispositivos modernos incorporan \textbf{Auto-MDI/MDI-X}, que detecta y adapta la configuración automáticamente.
\end{itemize}

\subsubsection{Estándares de cableado: T568A y T568B}
\label{sec:t568}

Al crimpar un cable Ethernet, cada uno de los 8 hilos del cable debe colocarse en una posición concreta del conector RJ-45. Existen dos estándares de asignación de colores:

\begin{center}
\renewcommand{\arraystretch}{1.2}
\begin{tabular}{c >{\bfseries}l >{\bfseries}l}
\toprule
\textbf{Pin} & \textbf{T568A} & \textbf{T568B} \\
\midrule
1 & \textcolor{green!70!black}{Blanco/Verde}   & \textcolor{orange}{Blanco/Naranja} \\
2 & \textcolor{green!70!black}{Verde}           & \textcolor{orange}{Naranja} \\
3 & \textcolor{orange}{Blanco/Naranja}          & \textcolor{green!70!black}{Blanco/Verde} \\
4 & \textcolor{blue}{Azul}                      & \textcolor{blue}{Azul} \\
5 & \textcolor{blue}{Blanco/Azul}               & \textcolor{blue}{Blanco/Azul} \\
6 & \textcolor{orange}{Naranja}                  & \textcolor{green!70!black}{Verde} \\
7 & \textcolor{brown}{Blanco/Marrón}            & \textcolor{brown}{Blanco/Marrón} \\
8 & \textcolor{brown}{Marrón}                   & \textcolor{brown}{Marrón} \\
\bottomrule
\end{tabular}
\end{center}

\begin{itemize}[nosep]
  \item \textbf{Cable directo:} ambos extremos con el \emph{mismo} estándar (T568B--T568B es lo más habitual en España).
  \item \textbf{Cable cruzado:} un extremo T568A y el otro T568B.
\end{itemize}

\subsection{Conectores de fibra óptica}

La fibra óptica transmite datos mediante \textbf{pulsos de luz} a través de hilos de vidrio o plástico. Ofrece velocidades muy superiores al cobre y es inmune a interferencias electromagnéticas. Aunque no es habitual encontrarla dentro de un PC doméstico, sí aparece en:
\begin{itemize}[nosep]
  \item La conexión FTTH (Fiber To The Home) que llega al router del hogar.
  \item Servidores, centros de datos y redes troncales.
  \item Tarjetas de red de fibra en equipos profesionales.
\end{itemize}

\begin{figure}[htbp]
  \centering
  \includegraphics[width=0.6\textwidth]{fibra}
  \caption{Conectores de fibra óptica más comunes: ST, SC, LC y FC.}
  \label{fig:fibra}
\end{figure}

\textbf{Conectores más comunes:}
\begin{itemize}[nosep]
  \item \textbf{LC (Lucent Connector):} pequeño (férula de 1,25~mm), de alta densidad. El más usado actualmente en centros de datos.
  \item \textbf{SC (Subscriber Connector):} férula de 2,5~mm, conexión por broche (push-pull). Muy utilizado en instalaciones FTTH.
  \item \textbf{ST (Straight Tip):} férula de 2,5~mm, conexión por bayoneta. Frecuente en redes de área local antiguas, en desuso.
\end{itemize}

\subsection{RJ-11 (telefonía)}

El conector \textbf{RJ-11 (6P2C/6P4C)} se utiliza para líneas telefónicas analógicas. Es más pequeño que el RJ-45 (6 posiciones frente a 8). Aunque la telefonía fija convencional está en retroceso, aún se encuentra en routers con puerto de teléfono y centralitas.

\begin{figure}[htbp]
  \centering
  \includegraphics[width=0.5\textwidth]{rj45vsrj11}
  \caption{Comparación de tamaño entre un conector RJ-45 (8 posiciones) y un RJ-11 (6 posiciones).}
  \label{fig:rj45_vs_rj11}
\end{figure}

\begin{warnbox}[No confundir RJ-11 y RJ-45]
Ambos conectores son transparentes y con clip, pero el RJ-11 es más \textbf{estrecho} (6 posiciones) que el RJ-45 (8 posiciones). Un conector RJ-11 \textbf{puede insertarse físicamente} en un puerto RJ-45, pero \textbf{no debe hacerse}: puede dañar los contactos del puerto RJ-45.
\end{warnbox}


% ══════════════════════════════════════════════════════════════════
\section{Conectores internos de datos: SATA y NVMe (M.2)}
\label{sec:disco}
% ══════════════════════════════════════════════════════════════════

\subsection{SATA III}

SATA (Serial ATA) es el estándar para conectar dispositivos de almacenamiento (HDD, SSD de 2,5'', unidades ópticas) a la placa base.

\begin{itemize}[nosep]
  \item \textbf{Conector de datos:} 7 pines, forma característica de ``I'' o ``L''. Cable delgado y flexible.
  \item \textbf{Conector de alimentación:} 15 pines, forma de ``L'' más ancha. Proporciona 3,3~V, 5~V y 12~V.
  \item \textbf{Velocidad máxima (SATA III):} 6 Gbps $\approx$ \textbf{600 MB/s} de tasa de transferencia real.
\end{itemize}

SATA~III es suficiente para discos duros mecánicos (HDD), que raramente superan los 200~MB/s. Sin embargo, los SSD SATA están \textbf{limitados por la interfaz}: un SSD SATA bueno lee a $\sim$550~MB/s, muy cerca del máximo teórico. Para superar esta barrera existe NVMe.

\subsection{NVMe sobre M.2}

\textbf{NVMe (Non-Volatile Memory Express)} es un protocolo de comunicación diseñado específicamente para SSD, que se comunica directamente con el procesador a través del bus \textbf{PCIe} (PCI Express), sin pasar por la controladora SATA.

El conector \textbf{M.2} es una ranura en la placa base donde se inserta directamente el SSD (sin cables). Es compacto y disponible en varios tamaños:

\begin{center}
\renewcommand{\arraystretch}{1.25}
\begin{tabular}{>{\bfseries}l l l}
\toprule
\textbf{Formato} & \textbf{Dimensiones} & \textbf{Uso} \\
\midrule
M.2 2230 & 22 $\times$ 30 mm & Portátiles ultradelgados, consolas (Steam Deck, PS5). \\
M.2 2242 & 22 $\times$ 42 mm & Algunos portátiles. \\
M.2 2260 & 22 $\times$ 60 mm & Poco frecuente. \\
M.2 2280 & 22 $\times$ 80 mm & \textbf{El más habitual} en placas de sobremesa y portátiles estándar. \\
\bottomrule
\end{tabular}
\end{center}

\begin{warnbox}[Ojo: no todo M.2 es NVMe]
Una ranura M.2 puede funcionar con interfaz \textbf{SATA} (máx. 600~MB/s) o con interfaz \textbf{PCIe/NVMe} (miles de MB/s). El tipo de \textbf{muesca (key)} del conector lo determina:
\begin{itemize}[nosep]
  \item \textbf{Key B:} SATA o PCIe $\times$2.
  \item \textbf{Key M:} PCIe $\times$4 (NVMe). Es el habitual en SSD de alto rendimiento.
  \item \textbf{Key B+M:} compatible con ambos (generalmente SATA).
\end{itemize}
Antes de comprar un SSD M.2, hay que comprobar qué interfaz soporta la ranura de la placa base.
\end{warnbox}

\subsubsection{Velocidades según generación PCIe}

\begin{center}
\renewcommand{\arraystretch}{1.25}
\begin{tabular}{>{\bfseries}l c l}
\toprule
\textbf{Generación PCIe} & \textbf{Velocidad máx.\ (x4)} & \textbf{Ejemplo de uso} \\
\midrule
PCIe Gen 3 & $\sim$3.500 MB/s  & SSD NVMe de generaciones 2018--2020. \\
PCIe Gen 4 & $\sim$7.000 MB/s  & SSD NVMe actuales de gama media-alta. \\
PCIe Gen 5 & $\sim$14.000 MB/s & SSD de última generación (2024+). \\
\bottomrule
\end{tabular}
\end{center}

\begin{keybox}[Comparativa rápida: HDD vs SSD SATA vs SSD NVMe]
\begin{center}
\begin{tabular}{l c c c}
\toprule
 & \textbf{HDD (SATA)} & \textbf{SSD (SATA)} & \textbf{SSD (NVMe Gen~4)} \\
\midrule
Lectura secuencial & $\sim$150 MB/s & $\sim$550 MB/s & $\sim$7.000 MB/s \\
Conexión & Cable SATA & Cable SATA & Ranura M.2 (sin cables) \\
\bottomrule
\end{tabular}
\end{center}
\end{keybox}


% ══════════════════════════════════════════════════════════════════
\section{Conectores de alimentación}
\label{sec:alimentacion}
% ══════════════════════════════════════════════════════════════════

Los conectores de la fuente de alimentación (PSU) distribuyen la energía a todos los componentes. Es crítico conocerlos para montar un equipo correctamente.

\begin{figure}[htbp]
  \centering
  \includegraphics[width=0.55\textwidth]{fuentealimentacion}
  \caption{Fuente de alimentación ATX con sus distintos cables y conectores.}
  \label{fig:psu}
\end{figure}

\subsection{Conectores principales (placa base)}

\begin{figure}[htbp]
  \centering
  \includegraphics[width=0.45\textwidth]{CONECTORATEX24P}
  \caption{Conector ATX de 24 pines (20+4) para alimentación principal de la placa base.}
  \label{fig:atx24}
\end{figure}

\begin{center}
\renewcommand{\arraystretch}{1.3}
\begin{tabularx}{\textwidth}{>{\bfseries}l c X}
\toprule
\textbf{Conector} & \textbf{Pines} & \textbf{Función} \\
\midrule
ATX 24 pines & 20+4 & Alimentación \textbf{principal} de la placa base. El conector de 20+4 permite compatibilidad con placas antiguas de 20 pines (se separan los 4 extra). Proporciona 3,3~V, 5~V y 12~V. \\
EPS / ATX12V & 4+4 (8 pines) & Alimentación específica del \textbf{procesador (CPU)}. Se conecta cerca del socket. El formato 4+4 permite usar solo 4 pines en placas que no requieran los 8. Proporciona 12~V. \\
\bottomrule
\end{tabularx}
\end{center}

\subsection{Conectores para tarjetas gráficas (PCIe / PEG)}

Las tarjetas gráficas potentes necesitan alimentación adicional más allá de los 75~W que proporciona la ranura PCIe de la placa.

\begin{figure}[htbp]
  \centering
  \includegraphics[width=0.4\textwidth]{CONECTORPEG6P}
  \caption{Conector PEG de 6 pines para alimentación de tarjetas gráficas. El formato 6+2 añade 2 pines separables.}
  \label{fig:peg}
\end{figure}

\begin{center}
\renewcommand{\arraystretch}{1.3}
\begin{tabularx}{\textwidth}{>{\bfseries}l c c X}
\toprule
\textbf{Conector} & \textbf{Pines} & \textbf{Potencia} & \textbf{Notas} \\
\midrule
PCIe 6 pines      & 6       & 75 W   & Gráficas de gama baja/media. \\
PCIe 6+2 pines    & 8 (6+2) & 150 W  & El más habitual. El formato ``6+2'' permite usarlo como 6 o como 8 pines según lo requiera la gráfica. \\
\textbf{12VHPWR}  & \textbf{16}  & \textbf{hasta 600 W} & Conector \textbf{nuevo} introducido con la especificación ATX~3.0 (2022). Usado por gráficas NVIDIA serie 40xx y posteriores. Incluye 4 pines de señal que informan a la fuente de la potencia demandada. \\
12V-2x6           & 16      & hasta 600 W & Revisión del 12VHPWR con mejoras mecánicas en el diseño del conector para evitar problemas de conexión (sobrecalentamiento si no se inserta completamente). \\
\bottomrule
\end{tabularx}
\end{center}

\begin{warnbox}[El conector 12VHPWR debe insertarse completamente]
Se han reportado casos de derretimiento del conector 12VHPWR cuando no se inserta a fondo. Es fundamental \textbf{verificar que el conector hace clic} y no doblar el cable en un ángulo cerrado cerca del conector. La revisión 12V-2x6 mejora este aspecto mecánico.
\end{warnbox}

\subsection{Conectores auxiliares}

\begin{center}
\renewcommand{\arraystretch}{1.3}
\begin{tabularx}{\textwidth}{>{\bfseries}l c X}
\toprule
\textbf{Conector} & \textbf{Pines} & \textbf{Función} \\
\midrule
SATA alimentación & 15 & Alimenta discos HDD, SSD de 2,5'' y unidades ópticas. Proporciona 3,3~V, 5~V y 12~V. Forma de ``L''. \\
Molex 4 pines     & 4  & Conector antiguo pero aún presente en algunas fuentes. Alimenta ventiladores adicionales, tiras LED, y mediante adaptadores puede alimentar dispositivos SATA. Proporciona 5~V y 12~V. \\
Ventilador (FAN)  & 3 o 4 & Se conecta a la placa base. 3 pines = voltaje variable. \textbf{4 pines = control PWM} (modulación por ancho de pulso), que permite a la placa ajustar la velocidad del ventilador con precisión según la temperatura. \\
\bottomrule
\end{tabularx}
\end{center}


% ══════════════════════════════════════════════════════════════════
\section{Conectores internos de la placa base: panel frontal y headers}
\label{sec:panelfrontal}
% ══════════════════════════════════════════════════════════════════

Al montar un PC, además de conectar alimentación y datos, es imprescindible cablear los \textbf{conectores del panel frontal de la caja} y los \textbf{headers internos} de la placa base. Son los conectores que más confusión generan en un montaje real.

\subsection{Conectores del panel frontal (Front Panel Header)}

La caja del PC tiene botones y LEDs en su parte delantera. Los cables que salen de la caja deben conectarse al \textbf{header de panel frontal} (\texttt{F\_PANEL} o \texttt{JFP1}) de la placa base. Son conectores diminutos, generalmente de 1 o 2 pines, sin clave de orientación.

\begin{center}
\renewcommand{\arraystretch}{1.3}
\begin{tabularx}{\textwidth}{>{\bfseries}l c X}
\toprule
\textbf{Conector} & \textbf{Pines} & \textbf{Función} \\
\midrule
Power SW     & 2 & \textbf{Botón de encendido.} Cierra un circuito momentáneamente. No tiene polaridad (da igual la orientación). \\
Reset SW     & 2 & \textbf{Botón de reinicio.} Funciona igual que Power SW. Sin polaridad. \\
Power LED +/-- & 2 & LED que indica si el equipo está encendido. \textbf{Tiene polaridad:} el cable marcado con ``+'' debe ir al pin ``+'' del header. \\
HDD LED +/--   & 2 & LED que parpadea cuando hay actividad en el disco. \textbf{Tiene polaridad.} \\
Speaker        & 4 (se usan 2) & Pequeño altavoz interno para los pitidos de la BIOS (\emph{beep codes}). Útil para diagnóstico cuando no hay imagen. \\
\bottomrule
\end{tabularx}
\end{center}

\begin{warnbox}[Consejo práctico: siempre consulta el manual]
La disposición de los pines en el header del panel frontal varía según el fabricante y modelo de placa base. \textbf{Siempre consulta el diagrama del manual} de la placa (o la serigrafía impresa sobre la propia placa). Si se conecta un LED al revés simplemente no se enciende; no se daña nada. Los botones (Power SW, Reset SW) funcionan en cualquier orientación.
\end{warnbox}

\subsection{Headers internos de USB}

Los puertos USB del panel frontal de la caja no están conectados directamente a la placa: necesitan cablearse a los \textbf{headers internos de USB}.

\begin{center}
\renewcommand{\arraystretch}{1.3}
\begin{tabularx}{\textwidth}{>{\bfseries}l c X}
\toprule
\textbf{Header} & \textbf{Pines} & \textbf{Descripción} \\
\midrule
USB 2.0 & 9 (2$\times$5 menos 1) & Header de dos filas, con un pin cegado para evitar conexión incorrecta. Cada header ofrece 2 puertos USB 2.0 frontales. Conector marcado como \texttt{USB\_2\_3} o similar. \\
USB 3.0 & 19 (2$\times$10 menos 1) & Header más grande y ancho, de color azul, con muesca lateral de guía. Ofrece 2 puertos USB 3.0 frontales. Conector marcado como \texttt{USB3\_1\_2}. \\
USB-C frontal & 20 & Header específico para el puerto USB-C frontal de cajas modernas. Más compacto, con clip de retención. Presente en placas de gama media-alta. \\
\bottomrule
\end{tabularx}
\end{center}

\subsection{Header de audio frontal (HD Audio)}

El panel frontal de la caja suele incluir un jack de auriculares y/o micrófono. Estos se conectan al header \texttt{HD\_AUDIO} (o \texttt{AAFP}) de la placa base.

\begin{itemize}[nosep]
  \item Conector de \textbf{9 pines} (2$\times$5 menos 1) con un pin cegado como guía.
  \item Estándar actual: \textbf{HD Audio} (Intel High Definition Audio, también conocido como Azalia).
  \item Estándar antiguo: \textbf{AC'97} (incompatible con HD Audio). Algunas cajas antiguas traen ambos conectores etiquetados.
\end{itemize}

\begin{infobox}[¿Y si el audio frontal no funciona?]
Si tras montar el PC los puertos de audio frontales no funcionan, lo primero es comprobar que el header \texttt{HD\_AUDIO} está correctamente conectado. Si la caja solo tiene el conector AC'97, puede ser necesario cambiar la configuración en la BIOS o en el software de audio de la placa (Realtek, etc.) para activar el modo AC'97.
\end{infobox}


% ══════════════════════════════════════════════════════════════════
\section{Conectividad inalámbrica}
\label{sec:wireless}
% ══════════════════════════════════════════════════════════════════

Aunque no son ``conectores físicos'' en sentido estricto, las tecnologías inalámbricas forman parte esencial de la conectividad de un equipo moderno. Las dos principales en un PC son \textbf{Wi-Fi} y \textbf{Bluetooth}.

\subsection{Wi-Fi (IEEE 802.11)}

Permite la conexión a redes locales sin cables. Los equipos pueden incorporar Wi-Fi mediante un \textbf{chip integrado en la placa base}, una \textbf{tarjeta PCIe interna} con antena, o un \textbf{adaptador USB} externo.

\begin{figure}[htbp]
  \centering
  \includegraphics[width=0.45\textwidth]{Wifi}
  \caption{Adaptador Wi-Fi con antena y puerto de conexión coaxial para la antena externa.}
  \label{fig:wifi}
\end{figure}

\begin{center}
\renewcommand{\arraystretch}{1.3}
\begin{tabularx}{\textwidth}{>{\bfseries}l l c c X}
\toprule
\textbf{Generación} & \textbf{Estándar} & \textbf{Banda} & \textbf{Vel.\ máx.} & \textbf{Notas} \\
\midrule
Wi-Fi 4 & 802.11n  & 2,4 / 5 GHz & 600 Mbps & Introducción de MIMO. \\
Wi-Fi 5 & 802.11ac & 5 GHz       & 3,5 Gbps & MU-MIMO. Aún muy extendido. \\
Wi-Fi 6 & 802.11ax & 2,4 / 5 GHz & 9,6 Gbps & OFDMA, mejor rendimiento en entornos congestionados. \\
Wi-Fi 6E & 802.11ax & 2,4 / 5 / \textbf{6 GHz} & 9,6 Gbps & Añade la banda de 6 GHz (menos saturada). \\
Wi-Fi 7 & 802.11be & 2,4 / 5 / 6 GHz & $\sim$46 Gbps & MLO (uso simultáneo de varias bandas). Estándar más reciente (2024). \\
\bottomrule
\end{tabularx}
\end{center}

\subsection{Bluetooth}

Tecnología inalámbrica de \textbf{corto alcance} (hasta $\sim$100 m en condiciones ideales) que opera en la banda de \textbf{2,4~GHz}. Se utiliza para conectar periféricos sin cable: auriculares, ratones, teclados, mandos de juego, altavoces\ldots

\begin{itemize}[nosep]
  \item \textbf{Bluetooth 4.0+:} introdujo \textbf{BLE (Bluetooth Low Energy)}, esencial para wearables (pulseras de actividad, relojes inteligentes) y dispositivos IoT.
  \item \textbf{Bluetooth 5.0:} duplica el alcance y cuadruplica la velocidad respecto a 4.0. Permite transmitir audio a dos dispositivos simultáneamente.
  \item \textbf{Bluetooth 5.3/5.4:} mejoras en eficiencia energética y estabilidad de conexión.
\end{itemize}

En un PC de sobremesa, el Bluetooth suele venir \textbf{integrado con la tarjeta Wi-Fi} (tarjeta combo Wi-Fi + BT por PCIe o M.2) o puede añadirse con un \textbf{dongle USB Bluetooth}.

\subsection{Otras tecnologías inalámbricas (mención breve)}

\begin{figure}[htbp]
  \centering
  \begin{minipage}[t]{0.45\textwidth}
    \centering
    \includegraphics[width=0.7\textwidth]{MODULOZIGBEE}
    \caption{Módulo ZigBee para redes de sensores y domótica.}
    \label{fig:zigbee}
  \end{minipage}
  \hfill
  \begin{minipage}[t]{0.45\textwidth}
    \centering
    \includegraphics[width=0.7\textwidth]{Lora}
    \caption{Módulo LoRa para comunicaciones IoT de largo alcance.}
    \label{fig:lora}
  \end{minipage}
\end{figure}

\begin{itemize}[nosep]
  \item \textbf{NFC (Near Field Communication):} comunicación a menos de 10~cm. Pagos móviles, identificación.
  \item \textbf{ZigBee (IEEE 802.15.4):} bajo consumo, baja velocidad (250~kbps), redes de sensores y domótica.
  \item \textbf{LoRa (Long Range):} largo alcance (hasta 20~km), muy baja velocidad. IoT rural, medidores inteligentes.
\end{itemize}


% ══════════════════════════════════════════════════════════════════
\section{Conectores en desuso: referencia histórica}
\label{sec:obsoletos}
% ══════════════════════════════════════════════════════════════════

Los siguientes conectores ya no se montan en equipos nuevos, pero es útil conocerlos porque pueden aparecer en equipos antiguos que necesiten mantenimiento.

\begin{center}
\renewcommand{\arraystretch}{1.3}
\begin{tabularx}{\textwidth}{>{\bfseries}l l l X}
\toprule
\textbf{Conector/Puerto} & \textbf{Tipo} & \textbf{Uso original} & \textbf{Sustituido por} \\
\midrule
PS/2 (Mini-DIN 6)     & Externo & Ratón (verde) y teclado (violeta) & USB \\
Puerto serie (COM, DE-9M) & Externo & Ratón, módem, consola de routers & USB (aunque la consola serie persiste vía adaptador USB-serie) \\
Puerto paralelo (DB-25F)  & Externo & Impresoras, escáneres     & USB \\
VGA (DE-15F)              & Vídeo   & Monitores CRT y primeros LCD & HDMI / DisplayPort \\
DVI                       & Vídeo   & Monitores LCD (2000s)      & HDMI / DisplayPort \\
FireWire (IEEE 1394)      & Externo & Cámaras DV, discos externos & USB 3.x / Thunderbolt \\
IDE (ATA, 40 pines)       & Interno & HDD, unidades ópticas      & SATA \\
SCSI                      & Interno & Servidores, estaciones de trabajo & SAS / NVMe \\
Molex (datos disquetera, 34 pines) & Interno & Disqueteras & Eliminado \\
RCA (vídeo/audio)         & Ext.    & TV, reproductores DVD      & HDMI \\
\bottomrule
\end{tabularx}
\end{center}

\begin{figure}[htbp]
  \centering
  \begin{minipage}[t]{0.3\textwidth}
    \centering
    \includegraphics[width=\textwidth]{PUERTOPS2}
    \caption{Puertos PS/2.}
    \label{fig:ps2}
  \end{minipage}
  \hfill
  \begin{minipage}[t]{0.3\textwidth}
    \centering
    \includegraphics[width=\textwidth]{PUERTOSERIE}
    \caption{Puerto serie (COM).}
    \label{fig:serie}
  \end{minipage}
  \hfill
  \begin{minipage}[t]{0.3\textwidth}
    \centering
    \includegraphics[width=\textwidth]{PUERTOPARALELO}
    \caption{Puerto paralelo.}
    \label{fig:paralelo}
  \end{minipage}
\end{figure}

\begin{figure}[htbp]
  \centering
  \begin{minipage}[t]{0.3\textwidth}
    \centering
    \includegraphics[width=\textwidth]{VGA}
    \caption{Conector VGA.}
    \label{fig:vga}
  \end{minipage}
  \hfill
  \begin{minipage}[t]{0.3\textwidth}
    \centering
    \includegraphics[width=\textwidth]{CONECTORDVI}
    \caption{Conector DVI.}
    \label{fig:dvi}
  \end{minipage}
  \hfill
  \begin{minipage}[t]{0.3\textwidth}
    \centering
    \includegraphics[width=\textwidth]{ConectoresFIREWIRE}
    \caption{Conectores FireWire.}
    \label{fig:firewire}
  \end{minipage}
\end{figure}

\begin{infobox}[¿Por qué conocer los conectores antiguos?]
En un ciclo de Montaje y Mantenimiento, es habitual encontrar equipos con varios años de antigüedad. Saber identificar un puerto serie o un conector IDE permite diagnosticar y mantener estos equipos, o saber qué adaptador se necesita para hacerlos funcionar con periféricos modernos.
\end{infobox}


% ══════════════════════════════════════════════════════════════════
\section{Resumen visual: conectores que debes conocer hoy}
\label{sec:resumen}
% ══════════════════════════════════════════════════════════════════

\begin{center}
\renewcommand{\arraystretch}{1.35}
\begin{tabularx}{\textwidth}{>{\bfseries}l l l X}
\toprule
\textbf{Conector} & \textbf{Tipo} & \textbf{Velocidad/Capacidad} & \textbf{Para qué lo usarás} \\
\midrule
USB-A 3.x       & Externo & 5--10 Gbps       & Periféricos, discos externos, pendrives \\
USB-C            & Externo & Hasta 80 Gbps    & Todo: datos, vídeo, carga, audio \\
Thunderbolt 4/5  & Externo & 40--120 Gbps     & Docking stations, eGPU, monitores \\
HDMI 2.1         & Vídeo   & 48 Gbps          & TV, consolas, monitores \\
DisplayPort 2.1  & Vídeo   & 77,4 Gbps        & Monitores PC, gaming, multi-monitor \\
Jack 3,5~mm      & Audio   & Analógico         & Auriculares, altavoces, micrófono \\
RJ-45            & Red     & 1--10 Gbps        & Red cableada Ethernet \\
SATA III         & Interno & 600 MB/s          & HDD, SSD 2,5'' \\
M.2 NVMe         & Interno & Hasta 14.000 MB/s & SSD de alto rendimiento \\
ATX 24 pines     & Alim.   & ---               & Alimentación de placa base \\
EPS 4+4          & Alim.   & ---               & Alimentación de CPU \\
PCIe 6+2         & Alim.   & ---               & Alimentación de gráfica \\
12VHPWR/12V-2x6  & Alim.   & ---               & Gráficas de nueva generación \\
SATA alim. 15p   & Alim.   & ---               & Alimentación de discos y ópticas \\
FAN 4 pines (PWM)& Alim.   & ---               & Ventiladores con control de velocidad \\
F\_PANEL         & Interno & ---               & Botones y LEDs del panel frontal \\
USB 3.0 header   & Interno & ---               & Puertos USB frontales de la caja \\
HD Audio header  & Interno & ---               & Jack de audio frontal de la caja \\
\bottomrule
\end{tabularx}
\end{center}


% ══════════════════════════════════════════════════════════════════
\newpage
\section{Ejercicios}
\label{sec:ejercicios}
% ══════════════════════════════════════════════════════════════════

\subsection{Preguntas de repaso}

\begin{enumerate}[label=\textbf{R\arabic*.}, leftmargin=2em]

\item ¿Qué diferencia hay entre el \textbf{tipo de conector} USB (forma física) y el \textbf{estándar} USB (velocidad)? Pon un ejemplo de cómo puede llevar a confusión.

\item Un puerto USB-A de color azul, ¿qué velocidad teórica mínima garantiza? ¿Y si fuera de color negro?

\item Explica qué es \textbf{USB Power Delivery (USB PD)} y por qué es importante en la tendencia actual hacia el cargador universal.

\item ¿Qué símbolo identifica un puerto \textbf{Thunderbolt}? ¿Puede un dispositivo USB-C funcionar en un puerto Thunderbolt? ¿Y al revés?

\item Indica dos diferencias clave entre \textbf{HDMI} y \textbf{DisplayPort} que harían que recomendaras uno u otro según el escenario.

\item ¿Qué significan las siglas \textbf{VRR}, \textbf{ALLM} y \textbf{eARC} en el contexto de HDMI~2.1? Explica brevemente para qué sirve cada una.

\item ¿Qué diferencia hay entre un conector Jack \textbf{TRS} y uno \textbf{TRRS}? ¿En qué tipo de auriculares encontrarías cada uno?

\item ¿Por qué es importante elegir la \textbf{categoría de cable Ethernet} adecuada? ¿Qué categoría mínima recomendarías para una instalación nueva en una oficina?

\item Explica la diferencia entre un SSD con interfaz \textbf{SATA} y uno con interfaz \textbf{NVMe}. ¿Por qué el NVMe es más rápido?

\item ¿Qué es la \textbf{muesca (key)} de un conector M.2 y por qué es importante verificarla antes de comprar un SSD?

\item ¿Cuántos vatios puede suministrar la ranura PCIe de la placa base a una tarjeta gráfica sin alimentación adicional? ¿Por qué las gráficas potentes necesitan conectores PCIe adicionales?

\item ¿Qué problema específico se ha reportado con el conector \textbf{12VHPWR} y qué precaución debe tomarse al instalarlo?

\item Al montar un PC, ¿qué conectores del panel frontal \textbf{tienen polaridad} y cuáles no? ¿Qué ocurre si conectas un LED al revés?

\item ¿Qué diferencia hay entre el header de \textbf{USB 2.0 interno} (9 pines) y el de \textbf{USB 3.0 interno} (19 pines)? ¿Son intercambiables?

\end{enumerate}

\subsection{Casos prácticos}

\begin{enumerate}[label=\textbf{CP\arabic*.}, leftmargin=2em]

\item \textbf{Montaje de un equipo gaming:} Un cliente quiere un PC para jugar a 4K a 120~Hz. Está dudando entre un monitor HDMI y uno DisplayPort. También quiere un SSD que alcance al menos 5.000~MB/s de lectura.
\begin{enumerate}[label=\alph*)]
  \item ¿Qué versión mínima de HDMI necesitaría para 4K a 120~Hz?
  \item ¿Y de DisplayPort?
  \item ¿Qué tipo de SSD y conexión necesita para alcanzar esa velocidad?
  \item La tarjeta gráfica que ha elegido requiere un conector de 16 pines. ¿De qué conector se trata? ¿Qué fuente de alimentación necesita?
\end{enumerate}

\item \textbf{Estación de trabajo con docking station:} Una diseñadora gráfica quiere conectar su portátil a dos monitores 4K, un disco externo rápido, un teclado, un ratón y cargar el portátil, todo con \textbf{un solo cable}.
\begin{enumerate}[label=\alph*)]
  \item ¿Qué tecnología de conexión le recomendarías?
  \item ¿Qué versión mínima necesita para soportar dos monitores 4K a 60~Hz?
  \item ¿Qué tipo de docking station necesita?
\end{enumerate}

\item \textbf{Diagnóstico de red lenta:} En una pequeña empresa, la red Ethernet va lenta. El administrador descubre que están usando cables Cat~5e y un switch Gigabit, pero los usuarios que hacen edición de vídeo necesitan más ancho de banda.
\begin{enumerate}[label=\alph*)]
  \item ¿Es el cable Cat~5e capaz de soportar Gigabit Ethernet?
  \item Si necesitan 10 Gigabit Ethernet, ¿qué categoría mínima de cable deben instalar?
  \item ¿Necesitarían cambiar también el switch? ¿Por qué?
\end{enumerate}

\item \textbf{Compatibilidad de conectores:} Un alumno está montando un PC y se encuentra con las siguientes situaciones. Indica si es posible o no, y qué se necesitaría:
\begin{enumerate}[label=\alph*)]
  \item Conectar un monitor DisplayPort a una salida HDMI de la tarjeta gráfica.
  \item Conectar un SSD M.2 con key~M a una ranura M.2 con key~B.
  \item Usar un cable USB-C para cargar un portátil que solo tiene puerto USB-A.
  \item Conectar una fuente con conector PCIe de 6+2 pines a una gráfica que pide 6 pines.
  \item Conectar unos auriculares con conector TRRS a una placa base que tiene puertos separados para micrófono (rosa) y auriculares (verde).
\end{enumerate}

\item \textbf{Montaje del panel frontal:} Has montado un PC nuevo pero al pulsar el botón de encendido no ocurre nada. El ventilador de la CPU no gira y no hay imagen.
\begin{enumerate}[label=\alph*)]
  \item ¿Cuál es la primera conexión que comprobarías?
  \item Si el PC enciende pero el LED de encendido no se ilumina, ¿a qué puede deberse? ¿Se ha dañado algo?
  \item Los puertos USB frontales de la caja no funcionan. ¿Qué header interno debes comprobar?
  \item El audio frontal (jack de auriculares) no produce sonido. Enumera dos posibles causas.
\end{enumerate}

\end{enumerate}

\subsection{Práctica de taller: crimpado de un cable Ethernet RJ-45}
\label{sec:practica_crimpado}

\begin{keybox}[Objetivo de la práctica]
Fabricar un \textbf{cable de red Ethernet directo} (patch cord) con conectores RJ-45, siguiendo el estándar \textbf{T568B} en ambos extremos, y verificar su correcto funcionamiento.
\end{keybox}

\textbf{Material necesario:}
\begin{multicols}{2}
\begin{itemize}[nosep]
  \item Cable UTP Cat~6 (1--2 metros)
  \item 2 conectores RJ-45
  \item Crimpadora para RJ-45
  \item Pelacables o tijeras
  \item Tester (comprobador) de cables de red
\end{itemize}
\end{multicols}

\textbf{Procedimiento:}
\begin{enumerate}
  \item \textbf{Pelar la cubierta:} retirar aproximadamente 2--3~cm de la cubierta exterior del cable, dejando expuestos los 4 pares trenzados. Cuidado de no dañar el aislamiento interior de los hilos.
  \item \textbf{Destrenzar y ordenar:} separar los 8 hilos y ordenarlos \textbf{de izquierda a derecha} según el estándar T568B (ver sección~\ref{sec:t568}):
  \begin{center}
    \textcolor{orange}{Bl/Naranja} -- \textcolor{orange}{Naranja} -- \textcolor{green!70!black}{Bl/Verde} -- \textcolor{blue}{Azul} -- \textcolor{blue}{Bl/Azul} -- \textcolor{green!70!black}{Verde} -- \textcolor{brown}{Bl/Marrón} -- \textcolor{brown}{Marrón}
  \end{center}
  \item \textbf{Cortar los hilos:} alinear las puntas y cortarlas rectas a unos 12--14~mm de la cubierta. Los hilos deben quedar paralelos y a la misma altura.
  \item \textbf{Insertar en el conector:} introducir los hilos en el conector RJ-45 con el clip hacia abajo, asegurándose de que:
  \begin{itemize}[nosep]
    \item Cada hilo llega hasta el final del conector (se ven las puntas de cobre al mirarlo de frente).
    \item La cubierta exterior del cable entra al menos 5~mm dentro del conector (para que la grapa de tensión la sujete).
  \end{itemize}
  \item \textbf{Crimpar:} insertar el conector en la crimpadora y presionar firmemente. Esto empuja las cuchillas metálicas del conector a través del aislamiento de cada hilo, haciendo contacto eléctrico.
  \item \textbf{Repetir} en el otro extremo del cable con el mismo orden T568B.
  \item \textbf{Verificar con el tester:} conectar ambos extremos al comprobador de cables. Deben encenderse los 8 LEDs en orden (1--8). Si alguno no enciende o se enciende fuera de orden, hay un hilo mal colocado o sin contacto.
\end{enumerate}

\textbf{Criterios de evaluación:}
\begin{itemize}[nosep]
  \item[$\square$] Los 8 hilos hacen contacto correctamente (tester OK).
  \item[$\square$] El orden de colores es T568B en ambos extremos.
  \item[$\square$] La cubierta exterior queda sujeta por la grapa del conector.
  \item[$\square$] El cable está limpio, sin hilos expuestos fuera del conector.
\end{itemize}

\subsection{Ejercicio de investigación}

\begin{enumerate}[label=\textbf{I\arabic*.}, leftmargin=2em]

\item Busca en la web las especificaciones de una placa base actual de gama media (por ejemplo, una ASUS, MSI o Gigabyte con chipset reciente). Lista todos los conectores y puertos que incluye en su panel trasero y en la placa, y clasifícalos según las categorías estudiadas en este tema.

\item La Unión Europea ha establecido el \textbf{USB-C como cargador universal obligatorio} desde 2024. Investiga: ¿a qué dispositivos afecta esta normativa? ¿Afecta a los portátiles? ¿Desde cuándo?

\item Compara las especificaciones de un \textbf{SSD NVMe PCIe Gen~4} y un \textbf{SSD NVMe PCIe Gen~5} reales (busca modelos concretos). ¿Merece la pena el salto generacional a día de hoy en precio/rendimiento?

\end{enumerate}

\subsection{Ejercicio de identificación}

\begin{enumerate}[label=\textbf{ID\arabic*.}, leftmargin=2em]

\item Tu profesor te entregará (o mostrará fotos de) varios cables y conectores. Para cada uno, debes identificar:
\begin{enumerate}[label=\alph*)]
  \item Nombre del conector.
  \item Si es macho o hembra.
  \item Tipo de señal que transporta (datos, vídeo, audio, alimentación).
  \item Estándar/versión si es posible determinarlo visualmente.
  \item Si está en uso actualmente o es obsoleto.
\end{enumerate}

\item Dibuja un esquema de la parte trasera de una torre de PC moderna y señala dónde esperarías encontrar cada uno de estos puertos: USB-A 3.0, USB-C, HDMI, DisplayPort, RJ-45, Jack 3,5~mm (verde, rosa, azul).

\end{enumerate}

% ══════════════════════════════════════════════════════════════════
\vfill
\begin{center}
\small\textcolor{gray}{---~Fin del resumen del Tema~5~---}
\end{center}

\end{document}
